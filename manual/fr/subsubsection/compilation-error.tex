\documentclass[manual-fr.tex]{subfiles}
\begin{document}
Wapiti est un logiciel ayant recours à certaines spécificités du matériel et
du système d'exploitation pour améliorer ses performances. En conséquence, il
est possible d'avoir des erreurs à l'étape I/3- étant dues à l'abscence de ces
spécificités sur votre machine.\\

La plus fréquente semble être dûe à la fonction "\_\_sync\_bool\_compare\_and\_swap"
présente dans le fichier "gradient.c". Si la commande make provoque une erreur
et vous affiche des messages relatifs à cette fonction, la procédure est très
simple.\\

Dans le fichier "wapiti.h", cherchez la ligne :\\
//\#define ATM\_ANSI\\
Et supprimez la chaîne "//" en début de ligne pour obtenir :\\
\#define ATM\_ANSI\\
Sauvegardez et reprenez la procédure d'installation.
\end{document}