\documentclass[manual-fr.tex]{subfiles}
\begin{document}
\SEM\ dispose de module indépendants les uns des autres, le programme principal faisant alors office d'aiguilleur vers le module à lancer.\\

Pour obtenir la liste des modules disponibles et la syntaxe générale pour les lancer :

\begin{lstlisting}[style=pieceofcode,frame=single]
python -m sem (--help ou -h)
\end{lstlisting}

Pour connaître la version de \SEM\ :

\begin{lstlisting}[style=pieceofcode,frame=single]
python -m sem (--version ou -v)
\end{lstlisting}

Pour connaître les informations relatives à la dernière version de \SEM\ :

\begin{lstlisting}[style=pieceofcode,frame=single]
python -m sem (--informations ou -i)
\end{lstlisting}


Pour lancer un module, la syntaxe générale est :

\begin{lstlisting}[style=pieceofcode,frame=single]
python -m sem $<$nom\_du\_module$>$ $<$arguments\_et\_options\_du\_module$>$
\end{lstlisting}

Les différents modules seront détaillés un par un.

\end{document}
