\documentclass[manual-fr.tex]{subfiles}
\begin{document}

\begin{itemize}
    \item[] \textbf{description}
        \begin{itemize}
            \item[] Compile (sérialise) un fichier dictionnaire qui pourra alors être utilisé en ressource dans \SEM.
        \end{itemize}
    \item[] \textbf{arguments}
        \begin{itemize}
            \item[] input : fichier dictionnaire
                \begin{itemize}
                    \item[] Le dictionnaire à compiler.
                \end{itemize}
            \item[] output : fichier compilé
                \begin{itemize}
                    \item[] Le dictionnaire compilé.
                \end{itemize}
        \end{itemize}
    \item[] \textbf{options}
        \begin{itemize}
            \item[] --help ou -h : switch
                \begin{itemize}
                    \item[] affiche l'aide
                \end{itemize}
            \item[] -k ou --kind : énumération \{token, multiword\}
                \begin{itemize}
                    \item[] le type de dictionnaire en entrée. token : chaque entrée représente un mot. multiword : chaque entrée représente une suite de mots.
                \end{itemize}
            \item[] -i ou --input-encoding : string
                \begin{itemize}
                    \item[] définit l'encodage du fichier d'entrée. Prioritaire sur la valeur de --encoding (défaut : --encoding).
                \end{itemize}
            \item[] --log ou -l : string
                \begin{itemize}
                    \item[] définit le niveau de log : info, warn ou critical (défaut : critical).
                \end{itemize}
            \item[] --log-file : fichier
                \begin{itemize}
                    \item[] le fichier où écrire le log (défaut : sortie terminal).
                \end{itemize}
        \end{itemize}
\end{itemize}

\end{document}