\documentclass[manual-fr.tex]{subfiles}
\begin{document}

\begin{itemize}
    \item[] \textbf{description}
        \begin{itemize}
            \item[] Transforme un texte au format vectorisé en un texte au format linéaire. Il est possible d'ajouter des informations de
				POS et / ou de chunking au texte linéaire.
        \end{itemize}
    \item[] \textbf{arguments}
        \begin{itemize}
            \item[] input : fichier vectorisé
                \begin{itemize}
                    \item[] le fichier d'entrée. Doit contenir au moins une annotation PoS ou une annotation en chunks.
                \end{itemize}
            \item[] output : fichier linéaire de sortie.
                \begin{itemize}
                    \item[] Ne contient que le texte ainsi que les informations PoS et / ou les informations de chunking.
                \end{itemize}
        \end{itemize}
    \item[] \textbf{options}
        \begin{itemize}
            \item[] --help ou -h : switch
                \begin{itemize}
                    \item[] affiche l'aide
                \end{itemize}
            \item[] -p ou --pos-column : entier
                \begin{itemize}
                    \item[] l'index où les informations PoS se trouve.
                \end{itemize}
            \item[] -c ou --chunk-column : entier
                \begin{itemize}
                    \item[] l'index où se trouvent les informations de chunking.
                \end{itemize}
            \item[] --input-encoding : string
                \begin{itemize}
                    \item[] définit l'encodage du fichier d'entrée. Prioritaire sur la valeur de --encoding (défaut : --encoding).
                \end{itemize}
            \item[] --output-encoding : string
                \begin{itemize}
                    \item[] définit l'encodage du fichier de sortie. Prioritaire sur la valeur de --encoding (défaut : --encoding).
                \end{itemize}
            \item[] --encoding : string
                \begin{itemize}
                    \item[] définit l'encodage du fichier d'entrée et de sortie. Si un encodage est fourni pour un fichier,
                        cette valeur est surchargée (défaut : UTF-8).
                \end{itemize}
            \item[] --log ou -l : string
                \begin{itemize}
                    \item[] définit le niveau de log : info, warn ou critical (défaut : critical).
                \end{itemize}
            \item[] --log-file : fichier
                \begin{itemize}
                    \item[] le fichier où écrire le log (défaut : sortie terminal).
                \end{itemize}
        \end{itemize}
\end{itemize}

\end{document}