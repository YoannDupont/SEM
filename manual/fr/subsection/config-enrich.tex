\documentclass[manual-fr.tex]{subfiles}
\begin{document}

Le fichier de configuration du module enrich permet d'ajouter des informations à un fichier vectorisé. Il décrit d'abord les entrées présentes
puis les informations à ajouter.\\

Le fichier d'enrichissement est un fichier XML de type de document ``enrich''. Il a trois parties : une ``entries'' qui définit les entrées déjà
présentes dans le fichier, une ``endogenous'' qui permet d'extraire des informations à partir des données présentes et une ``exogenous'' qui permet
d'intégrer des dictionnaires.\\

Chaque entrée (qu'elle soit déjà présente ou ajoutée) doit être nommée (via l'attribut \emph{``name''}) et deux entrées différentes ne peuvent pas
avoir le même nom. En gras sont les différentes ``sections'', en italique les différentes informations.\\

La liste des ``sections'' ainsi que les informations qu'il est possible d'ajouter est :

\begin{itemize}
	\item[] \textbf{entries}
	\begin{itemize}
		\item[] \textit{before}
			\begin{itemize}
				\item[] les entrées qui seront écrites avant les informations ajoutées
			\end{itemize}
		\item[] \textit{after}
			\begin{itemize}
				\item[] les entrées qui seront écrites après les informations ajoutées
			\end{itemize}
	\end{itemize}
	\item[] \textbf{endogenous}
	\begin{itemize}
		\item[] \textit{\{nullary, nary\}}
			\begin{itemize}
				\item[] des opérations génériques définies par leur arité (-ary).
			\end{itemize}
		\item[] \textit{expression}
			\begin{itemize}
				\item[] une information extraite à partir d'une expression booléenne.
			\end{itemize}
		\item[] \textit{list}
			\begin{itemize}
				\item[] une information extraite à partir d'une liste de mots et / ou d'expressions régulières.
			\end{itemize}
		\item[] \textit{regexp}
			\begin{itemize}
				\item[] une information extraite à partir d'une unique expression régulière.
			\end{itemize}
		\item[] \textit{conditional}
			\begin{itemize}
				\item[] une information avec déclencheur.
			\end{itemize}
	\end{itemize}
	\item[] \textbf{exogenous}
	\begin{itemize}
		\item[] \textit{token}
			\begin{itemize}
				\item[] un dictionnaire de mots.
			\end{itemize}
		\item[] \textit{multiword}
			\begin{itemize}
				\item[] un dictionnaire de séquences de mots.
			\end{itemize}
	\end{itemize}
\end{itemize}

\end{document}