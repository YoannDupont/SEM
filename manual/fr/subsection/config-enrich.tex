\documentclass[manual-fr.tex]{subfiles}
\begin{document}

Le fichier de configuration du module enrich permet d'ajouter des informations à un fichier vectorisé. Il décrit d'abord les entrées présentes
puis les informations à ajouter.\\

Le fichier d'enrichissement est un fichier XML de type de document ``enrich''. Il a trois parties : une ``entries'' qui définit les entrées déjà
présentes dans le fichier, une ``endogenous'' qui permet d'extraire des informations à partir des données présentes et une ``exogenous'' qui permet
d'intégrer des dictionnaires.\\

Chaque entrée (qu'elle soit déjà présente ou ajoutée) doit être nommée (via l'attribut \emph{``name''}) et deux entrées différentes ne peuvent pas
avoir le même nom. En gras sont les différentes ``sections'', en italique les différentes informations. Un exemple de fichier de configuration pour le module enrich est illustré dans la figure \ref{fig:sem-features}.\\

\begin{figure}[ht!]
\footnotesize
\begin{xml}
\xheader{xml version="1.0" encoding="UTF-8"}\\
\xmarker{information}{}{\\
  \xmarker{entries}{}{\\
    \xmarker{before}{}{\\
      \xunit{entry}{\xfield{name}{word}}\\
      \xunit{entry}{\xfield{name}{POS}}\\
    }\\
    \xmarker{after}{}{\\
      \xunit{entry}{\xfield{name}{NE} \xfield{mode}{train}}\\
    }\\
  }\\
  \xmarker{features}{}{\\
    \xunit{nullary}{\xfield{name}{lower} \xfield{action}{lower} \xfield{display}{no}}\\
    \xmarker{unary}{ \xfield{name}{starts-with-upper} \xfield{action}{isUpper}}{0}\\
    \xunit{dictionary}{\xfield{name}{title} \xfield{action}{token} \xfield{path}{title.txt} \xfield{entry}{lower}}\\
    \xmarker{find}{ \xfield{name}{VerbForward} \xfield{action}{forward} \xfield{return\_entry}{word}}{\\
      \xmarker{regexp}{ \xfield{action}{check} \xfield{entry}{POS}}{\^{}V}\\
    }\\
  }\\
}
\end{xml}
\caption{exemples de fichier de génération de features de SEM, il est utilisé par le module enrich. Il permet de rajouter des descripteurs qui seront alors utilisés par les algorithmes par apprentissage automatique.}
\label{fig:sem-features}
\end{figure}

\end{document}