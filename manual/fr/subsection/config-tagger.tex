\documentclass[manual-fr.tex]{subfiles}
\begin{document}

Le fichier de configuration du module tagger est appelé le fichier de configuration maître. Il permet de définir une séquence de traitements
(modules) ainsi que des options globales aux différents modules qui seront lancés les uns après les autres.\\

Le fichier maître est un fichier XML de type de document ``master''. Il a deux parties : une ``pipeline'' qui est une séquences de modules et
une ``options'' qui permet de définir les options globales.\\

Les mots en gras sont les différentes sections du fichier de configuration maître, celles entre crochets sont les attributs des balises XML.

\begin{itemize}
	\item[] \textbf{pipeline}
	\begin{itemize}
		\item[] segmentation (si présent : doit être le premier module)
		\item[] enrich \textit{[config]}
		\item[] tagger \textit{[model]}
		\item[] clean\_info \textit{[to-keep]}
		\item[] textualise \textit{[pos, chunk]}
	\end{itemize}
	\item[] \textbf{options}
	\begin{itemize}
		\item[] encoding \textit{[input-encoding, output-encoding]} : encodage de l'entrée et des sorties.
		\item[] log \textit{[level, file]} : le niveau de log et le fichier de log (terminal si fichier non donné)
		\item[] clean
	\end{itemize}
\end{itemize}

\end{document}