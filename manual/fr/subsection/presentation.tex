\documentclass[manual-fr.tex]{subfiles}
\begin{document}
\SEMFull\ \cite{Tellier_demo2012} est un logiciel d'annotation syntaxique du français.\\

Il permet la segmentation de texte brut en phrases, elles-même découpées en unités lexicales, mais il est tout-à-fait en mesure de traiter un texte
pré-segmenté. Les unités multi-mots peuvent être gérées de deux manières différentes : soit comme une seule unité lexicale où chaque mot est relié
par le caractère '\_', soit comme une suite de mots ayant une annotation particulière précisant que les mots sont reliés entre eux et possèdent
globalement la même fonction syntaxique.\\

SEM propose deux niveaux d'annotation syntaxique : le premier est une annotation morpho-syntaxique de chaque unité lexicale du texte selon le jeu
d'étiquettes défini par Crabbé et Candito (TALN 2008). Le second est une annotation en chunks selon le modèle BIO (Begin In Out), le programme
permettant d'obtenir un étiquetage selon un chunking complet ou bien partiel, auquel cas il ne reconnaîtra que les groupes nominaux (le chunking
partiel étant soumis à des règles différentes que le chunking complet, l'un n'est donc pas un sous-ensemble de l'autre).\\

Toutes les commandes du manuel sont mises entre guillemets pour les distinguer clairement du reste du texte, mais elle doit être écrite sans eux.
\end{document}