\documentclass[manual-fr.tex]{subfiles}
\begin{document}

L'annotation PoS se base sur le jeu d'étiquettes défini par \cite{Crabbe08} :\\

\begin{minipage}{0.49\linewidth}
	\begin{itemize}
		\item[] ADJ : adjectif
		\item[] ADJWH : adjectif
		\item[] ADV : adverbe
		\item[] ADVWH : adverbe
		\item[] CC : conjonction de coordination
		\item[] CLO : clitique objet
		\item[] CLR : clitique réfléchi
		\item[] CLS : clitique sujet
		\item[] CS : conjonction de subordination
		\item[] DET : déterminant
		\item[] DETWH : déterminant 
		\item[] ET : mot étranger
		\item[] I : interjection
		\item[] NC : nom commun
		\item[] NPP : nom propre
	\end{itemize}
\end{minipage}
\begin{minipage}{0.49\linewidth}
	\begin{itemize}
		\item[] P : préposition 
		\item[] P+D : préposition 
		\item[] P+PRO : préposition 
		\item[] PONCT : ponctuation
		\item[] PREF : préfixe
		\item[] PRO : pronom 
		\item[] PROREL : pronom 
		\item[] PROWH : pronom 
		\item[] VINF : verbe à l'infinitif
		\item[] VPR : verbe au participe présent
		\item[] VPP : verbe au participe passé
		\item[] V : verbe à l'indicatif
		\item[] VS : verbe au subjonctif
		\item[] VIMP : verbe à l'impératif
	\end{itemize}
\end{minipage}

\end{document}