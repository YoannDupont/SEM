\documentclass[manual-fr.tex]{subfiles}
\begin{document}

\begin{itemize}
    \item[] \textbf{description}
        \begin{itemize}
            \item[] it is the main \SEM\ module. It allows to process files using a pipeline. Pipes are processes made by either a module or \Wapiti. The modules to use and their order is given in an XML configuration file called the master file.
        \end{itemize}
    \item[] \textbf{arguments}
        \begin{itemize}
            \item[] master: XML file
                \begin{itemize}
                    \item[] the master configuration file. Defines the pipeline and the options.
                \end{itemize}
            \item[] input\_file: file
                \begin{itemize}
                    \item[] the input file.
                \end{itemize}
        \end{itemize}
    \item[] \textbf{options}
        \begin{itemize}
            \item[] --help ou -h: switch
                \begin{itemize}
                    \item[] displays help
                \end{itemize}
            \item[] --output-directory ou -o: directory
                \begin{itemize}
                    \item[] the output directory (default: current working directory).
                \end{itemize}
        \end{itemize}
\end{itemize}

\end{document}
