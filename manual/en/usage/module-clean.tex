\documentclass[manual-fr.tex]{subfiles}
\begin{document}

\begin{itemize}
    \item[] \textbf{description}
        \begin{itemize}
            \item[] clean allows to remove columns that are not useful.
        \end{itemize}
    \item[] \textbf{arguments}
        \begin{itemize}
            \item[] infile: file
                \begin{itemize}
                    \item[] The input file. Follows the CoNLL-2003 format.
                \end{itemize}
            \item[] outfile: file
                \begin{itemize}
                    \item[] The output file.
                \end{itemize}
            \item[] ranges: string
                \begin{itemize}
                    \item[] columns to keep. The user can give either a number or a range. A range is a couple of number separated by a colon. Multiple ranges can be given, they have to be separated by a comma.
                \end{itemize}
        \end{itemize}
    \item[] \textbf{options}
        \begin{itemize}
            \item[] --help ou -h: switch
                \begin{itemize}
                    \item[] displays help
                \end{itemize}
            \item[] --input-encoding: string
                \begin{itemize}
                    \item[] The encoding of the input file. Has priority over --encoding (default: --encoding).
                \end{itemize}
            \item[] --output-encoding: string
                \begin{itemize}
                    \item[] The encoding of the output file. Has priority over --encoding (default: --encoding).
                \end{itemize}
            \item[] --encoding: string
                \begin{itemize}
                    \item[] Encoding of both the input and the output files. Does not have priority (default: UTF-8).
                \end{itemize}
            \item[] --log ou -l: string
                \begin{itemize}
                    \item[] the log level: info, warn or critical (default: critical).
                \end{itemize}
            \item[] --log-file: file
                \begin{itemize}
                    \item[] the file where to log (default: terminal).
                \end{itemize}
        \end{itemize}
\end{itemize}

\end{document}
