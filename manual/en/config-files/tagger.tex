\documentclass[manual-fr.tex]{subfiles}
\begin{document}

The tagger module configuration file is called the master file. It allows to defined pipelines and global options that apply to the input files and modules.

The master file is an XML file of document type "master". It has two parts: a \emph{pipeline} that is a list of modules and an "option" part that allows to define global options. An example of such a file is given in figure \ref{fig:sem-pipeline}.\\

\begin{figure}[ht!]
\footnotesize
\begin{xml}
\xheader{xml version="1.0" encoding="UTF-8"}\\
\xmarker{master}{}{\\
  \xmarker{pipeline}{}{\\
    \xunit{segmentation}{\xfield{tokeniser}{fr}}\\
    \xunit{enrich}{\xfield{informations}{pos.xml} \xfield{mode}{label}}\\
    \xunit{annotate}{\xfield{model}{models/POS} \xfield{field}{POS}}\\
    \xunit{clean}{\xfield{to-keep}{word,POS}}\\
    \xunit{enrich}{\xfield{informations}{NER.xml} \xfield{mode}{label}}\\
    \xunit{annotate}{\xfield{model}{models/NER} \xfield{field}{NER}}\\
    \xunit{clean}{\xfield{to-keep}{word,POS,NER}}\\
  }\\
\\
  \xmarker{options}{}{\\
    \xunit{encoding}{\xfield{input-encoding}{utf-8} \xfield{output-encoding}{utf-8}}\\
    \xunit{log}{\xfield{log\_level}{INFO}}\\
    \xunit{export}{\xfield{format}{html} \xfield{pos}{POS} \xfield{ner}{NER} \xfield{lang}{fr} \xfield{lang\_style}{default.css}}\\
  }\\
}
\end{xml}
\caption{Specification of a \SEM\ pipeline.}
\label{fig:sem-pipeline}
\end{figure}

\end{document}
