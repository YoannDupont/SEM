\documentclass[manual-fr.tex]{subfiles}
\begin{document}

Les \textit{boolean features} \textit{boolean} définissent des expressions booléennes. Un exemple de feature \textit{boolean} est donné dans la figure \ref{fig:feature-boolean}. Trois actions sont disponibles:
\begin{itemize}
    \item and: logical and. Takes two \textit{boolean features} arguments.
    \item or: logical or. Take two \textit{boolean features} arguments.
    \item not: logical not. Takes one \textit{boolean feature} argument.
\end{itemize}

\begin{figure}[ht!]
\footnotesize
\begin{xml}
\xmarker{boolean}{ \xfield{name}{StartsWithUpper-AndNot-FirstWordOfSentence} \xfield{action}{and}}{\\
    \xmarker{unary}{ \xfield{action}{isUpper}}{0}\\
    \xmarker{boolean}{ \xfield{action}{not}}{\\
        \xunit{nullary}{\xfield{action}{BOS}}\\
    }\\
}
\end{xml}
\caption{example of a \textit{boolean} feature.}
\label{fig:feature-boolean}
\end{figure}

\end{document}
