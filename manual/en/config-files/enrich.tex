\documentclass[manual-fr.tex]{subfiles}
\begin{document}

The enrich module configuration file allows to add informations to a vectorised file. It first describres entries that are present then the informations to add.

It is and XML file of the document type "enrich". It has two parts: an "entries" defining entries that are already present and a "features" one to enrich the file.\\

Each entry (present or added) has to have a name (using the \emph{name} attribute). Two different entries cannot have the same name. An example of an enrich module configuration file is given in the figure \ref{fig:sem-features}. Each entry has a mode allowing \SEM\ to only consider it in certain contexts. The \emph{train} mode allows to use an entry only when the aim is to train the model. The default mode, \emph{label}, considers the entry no matter the context.\\

Most features have the following attributes:
\begin{itemize}
    \item name="string": the name of the feature. Mandatory for root features.
    \item action="string": for features of the same nature (eg: evaluating a regular expression), select the kind of result or a different computation.
    \item x="integer": the shift according to the current token (default: 0).
    \item entry="string" : the entry used by the feature (default: word or token, otherwise the first feature defined in entries).
    \item display="(yes|no)": whether the feature should be displayed or not. It is possible to not display a feature used as a temporary result.
\end{itemize}

There are multiple kinds features according to the kind of results they compute or the arguments they take into account:
\begin{itemize}
    \item \emph{token features} : they process each token independently. There are two kinds of token features:
    \begin{itemize}
        \item \emph{string features}: they return a string
        \item \emph{boolean features}: they return a boolean
    \end{itemize}
    \item \emph{sequence features}: they process the whole sequence and return a sequence
\end{itemize}

\begin{figure}[ht!]
\footnotesize
\begin{xml}
\xheader{xml version="1.0" encoding="UTF-8"}\\
\xmarker{information}{}{\\
  \xmarker{entries}{}{\\
    \xmarker{before}{}{\\
      \xunit{entry}{\xfield{name}{word}}\\
      \xunit{entry}{\xfield{name}{POS}}\\
    }\\
    \xmarker{after}{}{\\
      \xunit{entry}{\xfield{name}{NE} \xfield{mode}{train}}\\
    }\\
  }\\
  \xmarker{features}{}{\\
    \xunit{nullary}{\xfield{name}{lower} \xfield{action}{lower} \xfield{display}{no}}\\
    \xunit{ontology}{\xfield{name}{NER-ontology} \xfield{path}{dictionaries/NER-ontology} \xfield{display}{no}}\\
    \xmarker{fill}{ \xfield{name}{NER-ontology-POS} \xfield{entry}{NER-ontology} \xfield{filler-entry}{POS}}{\\
      \xmarker{string}{ \xfield{action}{equal}}{O}\\
    }\\
    \xmarker{find}{ \xfield{name}{NounBackward} \xfield{action}{backward} \xfield{return\_entry}{word}}{\\
      \xmarker{regexp}{ \xfield{action}{check} \xfield{entry}{POS}}{\^{}N}\\
    }\\
    \xmarker{find}{ \xfield{name}{NounForward} \xfield{action}{forward} \xfield{return\_entry}{word}}{\\
      \xmarker{regexp}{ \xfield{action}{check} \xfield{entry}{POS}}{\^{}N}\\
    }\\
  }\\
}
\end{xml}
\caption{enrich module configuration file example. It allows to add descriptors that will be used for machine leaning.}
\label{fig:sem-features}
\end{figure}

\end{document}
